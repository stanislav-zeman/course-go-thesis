\documentclass[
  digital,
  color,
  oneside,
  nosansbold,
  nocolorbold,
  lof,
  lot,
]{fithesis4}
\usepackage[resetfonts]{cmap}
\usepackage[T1,T2A]{fontenc}
\usepackage[
  main=english,
  english
]{babel}

\thesissetup{
    date        = \the\year/\the\month/\the\day,
    university  = mu,
    faculty     = fi,
    type        = bc,
    department  = Department of Computer Science,
    author      = Stanislav Zeman,
    gender      = m,
    advisor     = {RNDr. Martin Ukrop, Ph.D.},
    title       = {Designing a Comprehensive Course on Go},
    TeXtitle    = {Designing a Comprehensive Course on Go},
    keywords    = {go, golang, course, lectures, exercises, homework},
    TeXkeywords = {go, golang, course, lectures, exercises, homework},
    abstract    = {
      This thesis deals with designing a new course aimed at teaching the Go programming language. The course is designed to be taught in the academic environment. This work also puts the Faculty of Informatics of Masaryk University into context, and its design reflects some of the shortcomings of local courses.
    },
    thanks      = {
      These are the acknowledgements for my thesis.
    },
    bib         = example.bib,
    facultyLogo = fithesis-fi,
}
\usepackage{makeidx}
\makeindex
\usepackage{paralist}
\usepackage{amsmath}
\usepackage{amsthm}
\usepackage{amsfonts}
\usepackage{url}
\usepackage{markdown}
\usepackage{listings}
\lstset{
  basicstyle      = \ttfamily,
  identifierstyle = \color{black},
  keywordstyle    = \color{blue},
  keywordstyle    = {[2]\color{cyan}},
  keywordstyle    = {[3]\color{olive}},
  stringstyle     = \color{teal},
  commentstyle    = \itshape\color{magenta},
  breaklines      = true,
}
\usepackage{floatrow}
\floatsetup[table]{capposition=top}
\usepackage[babel]{csquotes}

\begin{document}

\chapter{Introduction}

\markright{\textsc{Introduction}}
\addcontentsline{toc}{chapter}{Introduction}

This is the introduction of my thesis.

\chapter{The Go programming language}

\section{Characteristics}

\section{Usage}

\chapter{The current state at FI MUNI}

\section{Lack of up-to-date technologies}

\subsection{Rust}

\section{RedHat Go Course}

\section{Motivation for extending the course}

\subsection{Questionnaire}

\subsection{Interest}

\chapter{The course design}

This chapter gives high-level insight into the course design and the reasoning behind the decisions taken. Detailed explanations about particular lectures, exercises, and homework are given in the fourth chapter.

\section{Elementary information}

The course primarily focuses on teaching the Go programming language. The main goal of the course is to equip its students with the knowledge often required in practice. As a result, the course goes beyond just teaching the fundamentals of Go.

The course is intended to be taught throughout the span of 12 weeks. The first 11 weeks are regular lessons, and the last week is reserved for consultation and/or as a potential substitute class.

\subsection{Licensing}

The course and its materials are publicly available under the Creative Commons Share Alike (CC BY-SA 4.0 Deed) license. The motivation behind this is to keep the course in the public space. Anyone who would like to adapt or reuse this course can do so. There is no reason to keep anyone from using any of its materials, even for those with commercial intentions.

The original course is also open-sourced under this license, and as it is a share-alike license, the new course had to respect it. There was an option to migrate to the GPLv3 license, which is compatible, but in the end, the CC license was preferred.

\section{Overview}

The course materials are structurally split into 3 major categories: lectures, exercises, and homework, to be specific, each with a dedicated git repository. All of these categories are inspected in the following subsections. The last subsection of this chapter briefly talks about the final projects.

\subsection{Lectures}

The lecture slides are indubitably the most vital type of material. 

The first 6 lessons are adaptations of the original RedHat Go Course slides, which underwent major changes and restructuring. The changes were made to make the taught topics more cohesive and easier to follow, as some of the code samples were unnecessarily long or complex. Several lessons, specifically those on REST API and Concurrency, presented outdated principles that warranted updating.

The last 5 lectures are completely new and extend the course with numerous technologies and third-party libraries. The original course focused solely on using the standard library that is generally not leveraged in practice. The new slides were designed to compensate for this fact.

\subsubsection{Tooling}

I considered multiple tools for creating the lecture slides. It certainly had to be a markdown-oriented technology for multiple reasons. A plain-text solution was a necessity as all materials are versioned using git. This, in turn, eases collaboration with potential open-source contributors, which can further extend the course or fix any existing mistakes.

The original RedHat Go course used the Go Present tool. This solution, created by the Go team, implements an exceptional feature: it can run the Go code in your slides and immediately present the output. For solely this reason, I chose to stick with it as any other markdown-styled solution did not offer any viable alternative to this.

I also considered Marp as the runner-up. This tool is used throughout the Rust course previously mentioned in the second chapter. Compared to the Go Present tool, the main benefits include much better formatting capabilities, the possibility of exporting slides to multiple data formats like PDF or HTML, and a more straightforward syntax.

\subsection{Exercises}

The exercises are designed to be complementary assignments to lectures. The general feedback on the first course run, as discussed in the second chapter, was a lack of hands-on experience. These exercises were designed to compensate for that. They are, by no means, an equivalent substitute for traditional standard seminars. The students are expected to spend only up to 30 minutes solving them.

Each lecture is paired with a single exercise that can, in some cases, be split into up to two tasks. Both of these, in some way, relate to the topics covered in the lecture.

The exercises are not intended to be reviewed or evaluated in any way. However, they can provide useful on-site insight into the student's understanding of the covered topic, which can be reflected later. These exercises should also serve as the ground for answering potential follow-up questions.

\subsection{Homework}

The homework exercises are a form of individual work. Their goal is to prepare students for their final projects. The homework tasks depend on each other and demonstrate building an application from the ground up.

Starting from the third week, the assignments are published with a 14-day deadline followed by a 7-day review period. The review period is designed to give students feedback that they will later incorporate into their solutions.

Contrary to lecture exercises, homework exercises are intended to be reviewed and graded. In the end, they should make up half of the final grade.

\subsection{Projects}

The projects are the final part of the course and make up the second half of the final grade. The projects are estimated to take 50 to 60 hours for each student. The size of the projects should approximately match that estimation.

The project requirement can be fulfilled in 3 ways:
\begin{itemize}
 \item Implementing a solo project.
 \item Implementing a team project.
 \item Creating an open-source pull request.
\end{itemize}

Regarding team projects, their size should be linearly scaled with the number of team members. The recommended number is up to 4 students. This is strongly advised not to be exceeded as going above it generally negatively affects the team's performance.

It should be noted that the size expectations for the open-source contributions are much lower than those for its project alternatives. Searching for an open-source project and further involvement in the codebase are considered upon its evaluation. The pull requests do not need to be merged, as we cannot guarantee the maintainers' activity. One thing to note is that contributing to projects with low activity or without active maintainers should be avoided.

As part of a bonus lecture, the students are presented with a set of technologies/libraries they can leverage in their project implementations. These primarily serve as a form of motivation.

The general incentive is to make students come up with their own projects. Therefore, no "official" project assignments will be published. This was deeply considered as it may demotivate some students in the final part of the semester. However, I believe that the exploratory bonus lecture is sufficient for driving the student's projects even without predefined assignments.

\section{Limitations}

\subsection{Covered Topics}

This subsection describes some possible inefficiencies that were later perceived but not incorporated into the course because of the lack of time or space.

\subsubsection{Auth}

The course does not cover authorization or authentication in any way. It is up to debate if the course should cover this topic and, if so, to what extent. I believe that education in this regard, especially with practical application, is insufficient at our faculty. A dedicated course tackling topics like JWT, Cookies, and OAuth2 and their usage should exist instead.

\subsubsection{APIs}

The course currently covers designing and implementing REST APIs. This is still the standard for most service communication. However, during the past decade, multiple other approaches emerged that are often incorporated into systems in practice, notably gRPC and GraphQL. \\

gRPC is an implementation of the Remote Procedure Call (RPC) protocol by Google (g). It is often used for server-to-server communication as it provides type safety, lowers latency thanks to transmitting binary data using its custom format, and performs client/server code generation. \\

GraphQL (GQL) is a query language that was created at Facebook. It is built on top of standard REST API. GQL is often used for APIs that operate with complicated structured data. It removes the necessity to implement numerous endpoints and instead allows GQL queries to be sent to a single endpoint that parses the query and responds only to data specified in the query. Clients can fetch all required data using a single request. This way, the process saves time and network bandwidth.

\subsubsection{DevOps/Ops overlap}

Some of the lectures, namely Infrastructure and Observability, overlap with the DevOps/Ops field. Although the lectures provide Go examples and try to apply the general knowledge to Go, these topics are not Go specific and could be extracted to an additional course in the future. This would free up space for other topics, such as those previously mentioned.

\section{Catalog specification}

This chapter specifies the common catalog criteria used at the Masaryk University.

\subsection{Extent and Intensity}

The following expression notes the intensity of lectures, seminars, and homework, plus the type of completion, respectively.

(2/0/1) + 1 (colloquium)

\subsection{Prerequisities}

No hard requirements are enforced. However, students are expected to have a basic understanding of networking and virtualization.

\subsection{Enrolment limitations}

The course is intended to be run with up to 20 students. It can later be scaled to a higher number, but currently, the intent is to test the new course in practice and interact more with students, which does not scale well and would require opening a second seminar.

\subsection{Course objectives}

Students will understand the fundamentals of the Go programming language and its common uses in practice. They will acquire the required knowledge for entry-level Go developer positions while writing idiomatic Go code, and they will be capable of applying this knowledge to real-life projects.

\subsection{Learning outcomes}

Students will be able to:
\begin{itemize}
    \item Write idiomatic Go code.
    \item Understand the Go concurrency model.
    \item Inspect and optimize Go applications.
    \item Develop REST API services.
    \item Containerize and deploy their applications.
    \item Implement persistence leveraging SQL databases.
    \item Instrument applications with various types of telemetry.
\end{itemize}

\subsection{Syllabus}

\begin{enumerate}
    \item Introduction to Go
    \begin{itemize}
        \item Origins
        \item Characteristics
        \item Use cases
        \item Environment setup
        \item Hello World!
    \end{itemize}
    \item Fundamentals \#1
    \begin{itemize}
        \item Packages \& Visibility
        \item Variables
        \item Data types
        \item Control flow
        \item Functions
        \item Pointers
        \item Structures
    \end{itemize}
    \item Fundamentals \#2
    \begin{itemize}
        \item Interfaces
        \item Errors
        \item Arrays
        \item Slices
        \item Maps
        \item Range
    \end{itemize}
    \item Concurrency \& parallelism
    \begin{itemize}
        \item Goroutines
        \item Runtime
        \item Channels
        \item Select
        \item Related packages
    \end{itemize}
    \item Advanced \#1
    \begin{itemize}
        \item Generics
        \item Packages
        \item Testing
    \end{itemize}
    \item Advanced \#2
    \begin{itemize}
        \item Benchmarks
        \item Optimizations
        \item CGo
        \item Unsafe \& Reflect
    \end{itemize}
    \item REST APIs
    \begin{itemize}
        \item JSON
        \item HTTP
        \item REST API
        \item HTTP package
        \item Routers \& Web frameworks
        \item OpenAPI
        \item Templating
    \end{itemize}
    \item Containers
    \begin{itemize}
        \item Containerization
        \item Docker
        \item Kubernetes
    \end{itemize}
    \item Databases
    \begin{itemize}
        \item SQL
        \item RDBMSs
        \item Migrations
        \item sql
        \item sqlx
        \item sqlc
        \item GORM
    \end{itemize}
    \item Infrastructure
    \begin{itemize}
        \item CI/CD
        \item Infrastructure
        \item GCP
    \end{itemize}
    \item Observability
    \begin{itemize}
        \item Health
        \item Logs
        \item Metrics
        \item Tracing
        \item OpenTelemetry
    \end{itemize}
\end{enumerate}

\subsection{Literature}

\begin{itemize}
    \item O'Reilly: Learning Go \cite{oreilly-learning-go}
\end{itemize}

\subsection{Teaching methods}

In-person lectures with hands-on exercises and reviewed homework assignments.

\subsection{Assessment methods}

Multiple homework assignments (50 points) and a final project (50 points), including its defense. For a successful completion, 70 out of the 100 points are required.

\chapter{The course content}

The content is described in the same order as it is presented to students throughout the semester.

\section{Lecture 00: The course}

This lecture is designed solely to provide elementary information about the course, such as its requirements and outline.

\section{Lecture 01: Introduction}

\subsubsection{Outline}

\begin{itemize}
    \item Introduction to Go
    \begin{itemize}
        \item Origins
        \item Key characteristics \& comparisons
        \item Motivation
    \end{itemize}
    \item IDEs \& editors
    \item Installing Go
    \item Go
    \begin{itemize}
        \item Go executable
        \item Modules
        \item Packages
        \item Hello world
    \end{itemize}
\end{itemize}

\subsubsection{Sections}

\subsection{Exercise 01}

\section{Lecture 02: Fundamentals \#1}

\subsubsection{Outline}

\begin{itemize}
    \item Types
    \item Variables
    \item Control flow
    \item Functions
    \item User-defined data types
    \item Pointers
\end{itemize}

\subsubsection{Sections}

\subsection{Exercise 02}

\section{Lecture 03: Fundamentals \#2}

\subsubsection{Outline}

\begin{itemize}
    \item Interfaces
    \item Errors
    \item Arrays
    \item Slices
    \item Maps
    \item Range
\end{itemize}

\subsubsection{Sections}

\subsection{Exercise 03}

\section{Lecture 04: Concurrency \& Parallelism}

\subsubsection{Outline}

\begin{itemize}
    \item Goroutines
    \item Runtime
    \item Channels
    \item Select
    \item Packages
\end{itemize}

\subsubsection{Sections}

\subsection{Exercise 04}

\section{Lecture 05: Advanced \#1}

\subsubsection{Outline}

\begin{itemize}
    \item Generics
    \item Packages
    \item Testing
\end{itemize}

\subsubsection{Sections}

\subsection{Exercise 05}

\section{Homework 01}

\section{Lecture 06: Advanced \#2}

\subsubsection{Outline}

\begin{itemize}
    \item Benchmarks
    \item Profiling
    \item Optimizations
    \begin{itemize}
        \item References \& Values
        \item Maps
        \item Slices
        \item Loops
        \item Strings
    \end{itemize}
    \item CGo
    \item Unsafe
    \item Reflection
\end{itemize}

\subsubsection{Sections}

\subsection{Exercise 06}

\section{Lecture 07: REST APIs}

\subsubsection{Outline}

\begin{itemize}
    \item JSON
    \item HTTP
    \item REST API
    \item Go \& HTTP
    \begin{itemize}
        \item Standard library
        \item Third-party libraries
        \item Testing webservers
    \end{itemize}
    \item OpenAPI
    \item Templating
\end{itemize}

\subsubsection{Sections}

\subsection{Exercise 07}

\section{Homework 02}

\section{Lecture 08}

\subsubsection{Outline}

\begin{itemize}
    \item Containers
    \item Docker
    \begin{itemize}
        \item Desktop
        \item Hub
        \item Build
        \item Compose
    \end{itemize}
    \item Podman
    \item Testcontainers
    \item Kubernetes
\end{itemize}

\subsubsection{Sections}

\subsection{Exercise 08: Containers}

\section{Lecture 09: Databases}

\subsubsection{Outline}

\begin{itemize}
    \item SQL
    \item RDBMSs
    \item Database migrations
    \item Go \& SQL
    \begin{itemize}
        \item sql
        \item sqlx
        \item sqlc
        \item GORM
    \end{itemize}
\end{itemize}

\subsubsection{Sections}

\subsection{Exercise 09}

\section{Homework 03}

\section{Lecture 10: Infrastructure}

The motivation behind this lecture is to give students an insight into the CI/CD process and how this practice can be applied in the context of Go projects. The second goal is to give students a basic knowledge of how such projects can be deployed and maintained.

The students are expected to understand basic software engineering practices and elementary networking.

\subsubsection{Outline}

\begin{itemize}
    \item Linters
    \item CI/CD
    \begin{itemize}
        \item GitLab CI/CD
        \item GitHub Actions
    \end{itemize}
    \item Infrastructure
    \item Cloud
    \begin{itemize}
        \item GCP
    \end{itemize}
\end{itemize}

\subsubsection{Sections}

The first section covers linters and expands on the knowledge acquired in the introductory lesson, where the built-in go-vet linter was introduced. Most of this section is dedicated to the golangci-lint tool, a linter aggregator incorporated in many projects and is the most complex tool today. Together with formatters, these tools are necessary in the continuous integration process introduced in the following section.

The second section introduces the CI/CD process. The primary goal of this section is to provide students with the knowledge to be able to operate the CI/CD pipeline using both GitLab CI/CD and GitHub Actions.

As the faculty operates a self-hosted GitLab instance, almost all of the git-related work is done using GitLab. Consequently, most students do not interact with GitHub throughout their studies. The secondary goal of this chapter is to familiarize students with GitHub so they can join the open-source community, as most open-source software is developed on the GitHub platform.

The third section gives an introduction to infrastructure design. Introduces basic concepts such as proxying, load balancing, and application scaling. Two related technologies are introduced to leverage these concepts, namely Traefik Proxy and Caddy. This section aims to give students a basic understanding of how to provision basic infrastructure and tackle basic infrastructure design problems such as fault tolerance, scaling, or encryption.

The last section introduces all major cloud providers (Amazon, Microsoft, and Google) and further develops into Google Cloud Platform (GCP). As Google offers free credits thanks to the faculty program, it is a great opportunity for students to get their hands on operating cloud infrastructure. A set of basic services that GCP offers are introduced so students can leverage these in their follow-up projects.

\subsection{Exercise 10}

The exercise comprises two tasks. The second task is more of a demo in which the tutor will guide the students through the GCP Console.

As students generally have a basic understanding of CI, thanks to their previous experiences, it is not practiced at the seminar. Instead, the seminar time is given to the GCP. The goal of the demo/exercise is to get students familiar with the GCP so they can provision their own infrastructure.

Tasks:
\begin{itemize}
    \item The first task involves setting up a Caddy server instance as a reverse proxy/load balancing multiple ping-pong service instances.
    \item The second task involves setting up an automatic build and deployment using Cloud Run, provisioning a Compute Engine virtual machine and a Cloud SQL SQL server that can later be used for the student's projects.
\end{itemize}
    
\section{Lecture 11: Observability}

\subsubsection{Outline}

\begin{itemize}
    \item Health
    \item Metrics
    \begin{itemize}
        \item Prometheus
        \item Grafana
    \end{itemize}
    \item Logs
    \begin{itemize}
        \item Loggers
        \item Loki
    \end{itemize}
    \item Traces
    \begin{itemize}
        \item Jaeger
    \end{itemize}
    \item OpenTelemetry
\end{itemize}

\subsubsection{Sections}

\subsection{Exercise 11}

\section{Homework 04}

\section{Projects}

\subsubsection{Outline}

\begin{itemize}
    \item CLI
    \item TUI
    \item Desktop
    \item Web
\end{itemize}

\chapter{Conclusion}

\printbibliography[heading=bibintoc]

\appendix
\chapter{Course materials}

All course materials are maintained under the \href{https://github.com/course-go}{course-go} GitHub organization. To preserve the current state of the materials, all relevant repositories were exported and compressed into an attachment to this thesis. Each top-level directory in the attached archive maps to a GitHub repository. The directories contain versions of materials that existed when submitting this thesis and are most likely out-of-date as you read this text. Inspect the respective GitHub repositories if you want to view the updated version.

The GitHub repositories for respective directories can be found under the following URLs:
\begin{itemize}
    \item \href{https://github.com/course-go/lectures}{lectures}
    \item \href{https://github.com/course-go/exercises}{exercises}
    \item \href{https://github.com/course-go/homework}{homework}
    \item \href{https://github.com/course-go/code}{code}
    \item \href{https://github.com/course-go/course}{course}
    \item \href{https://github.com/course-go/ping-pong}{ping-pong}
\end{itemize}

\section{lectures}

The lectures directory contains all slides and their respective assets (images, code samples, videos, etc.). The supplied README in the directory guides you on how to run the slides.

\section{exercises}

The exercises directory contains all exercises with assignments and related code. Each exercise is located in a separate directory containing a README that specifies the assignment.

\section{homework}

The homework directory contains all homework assignments. Each homework assignment is located in its respective directory. The assignments are given in the form of a README.

\section{code}

The code directory contains source code that either implements solutions to some of the exercises or implements some form of functionality used throughout the exercises.
\section{course}

The course directory contains information that serves as the starting point when people want to learn more about the course.
\section{ping-pong}

The ping-pong directory contains a simple REST API project used for teaching deployments and their automation. It is split up from the code project as it simplifies the deployments.

\end{document}
